\documentclass[a4paper,10pt]{article}
\usepackage[utf8]{inputenc}
\usepackage{graphicx}
\usepackage{amsmath}
\usepackage{amssymb}
\usepackage{amsthm}
\usepackage{booktabs}
\usepackage{caption}
\usepackage{geometry}
%\usepackage{hyperref}
\usepackage{makeidx}
\usepackage{microtype}
\usepackage{subfig}
\usepackage{tabularx}
\usepackage{url}
\usepackage{varioref}
\usepackage{xcolor}
\usepackage{multicol}
\usepackage[italian]{babel}
\usepackage{mathtools}
\usepackage{booktabs}
\usepackage{multirow}
\usepackage{verbatim}
\usepackage{float}



\title{Consigli per le relazioni
\begin{large}
Dipartimento di Fisica E.Fermi - Università di Pisa
\end{large}}

\author{Di Ubaldo Gabriele}
\date{}

\begin{document}
\maketitle
\begin {figure}[H]
\begin{center}
\includegraphics[width=3cm]{/home/zerch/Documents/UNIPI/Fisica1/images/unipi.jpg}
\end{center}
\end{figure}
\section{Consigli per le relazioni}
\begin{enumerate}
 \item Cerca di dare una spiegazione per ogni fenomeno/misura/risultato. Non lasciare niente senza spiegare.
 \item Scrivi la probabilità che il $\chi^2$ aumenti ripetendo le misure usando fourmilab.ch
 \item Scrivi gli errori per ogni misura (puoi includere una footnote per dire che gli errori di misura sono quelli degli strumenti)
 \item L'ultima cifra significativa di qualunque risultato dovrebbe essere dello stesso ordine dell'incertezza: $9.81\pm0.31$ è sbagliato; $9.8\pm0.3$ è giusto perchè l'errore deve sempre essere arrotondato ad una cifra significativa!
 \item Scrivi che programma e che algoritmo (Marquardt-Levenberg) hai usato per fare il fit.
 \item Metti usepackage{fancyhdr}  e pagestyle{fancy} all'inizio.
 \item Includi sempre le unità di misura!
 \item Fai qualche grafico in scala log e bilog!
\end{enumerate}




\end{document}