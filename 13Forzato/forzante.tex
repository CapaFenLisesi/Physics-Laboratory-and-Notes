\documentclass[a4paper,10pt]{article}
\usepackage[utf8]{inputenc}
\usepackage{graphicx}
\usepackage{amsmath}
\usepackage{amssymb}
\usepackage{amsthm}
\usepackage{booktabs}
\usepackage{caption}
\usepackage{geometry}
%\usepackage{hyperref}
\usepackage{makeidx}
\usepackage{microtype}
\usepackage{subfig}
\usepackage{tabularx}
\usepackage{url}
\usepackage{varioref}
\usepackage{xcolor}
\usepackage{multicol}
\usepackage[italian]{babel}
\usepackage{mathtools}
\usepackage{booktabs}
\usepackage{multirow}
\usepackage{verbatim}
\usepackage{float}


\addto\captionsenglish{
  \renewcommand{\contentsname}%
    {Indice}%
}



\title{Laboratorio I: Oscillazioni forzate \\ Analisi del fenomeno della risonanza\\
\begin{large}
Dipartimento di Fisica E.Fermi - Università di Pisa
\end{large}}

\author{Di Ubaldo Gabriele}
\date{}

\begin{document}

\maketitle

\section{Introduzione}
\subsection{Teoria}
\textbf{Obiettivo:} Studiare il fenomeno della risonanza causato dall'applicazione di una forzzante ad un oscillatore armonico.
L'equazione differenziale che descrive il moto del sistema è la seguente:
\begin{equation}\label{hooke}
\ddot{\theta}+2\gamma\dot{\theta}+\omega^2_0\theta}=F_0\cos(\omega t)
\end{equation}
Dove $\omega_0$ è la velocità angolare propria del pendolo, $\omega$ è la pulsazione della forzante, $\gamma$ è proporzionale allo smorzamento.
La pulsazione del moto smorzato è data da:
\begin{equation}
\omega_s=\sqrt{\omega^2_0-\gamma^2}
\end{equation}
La soluzione all'equazione differenziale è la seguente:
\begin{equation}
\theta(t)=C_1 e^{-\gamma t}\cos(\omega_st+\phi_1)+C_2(\omega t+\phi_2)
\end{equation}
Il primo termine dopo un certo tempo $t>>1/\gamma$ può essere trascurato e quindi possiamo ottenere il valore di $C_2$:
\begin{equation}
C_2=\frac{F_0}{\sqrt{(\omega_0^2-\omega^2)^2+4\gamma^2\omega^2}}
\end{equation}

\subsection{Apparato sperimentale}
\begin{itemize}
\item{Pendolo dotato di smorzatore}
\item{Motore collegato al pendolo tramite due molle}
\item{Programma di acquisizione dati e controllo del motore}
\end{itemize}



\section{Esperimento}
\subsection{Acquisizione misure}
Abbiamo proceduto a prendere misure del pendolo semplice con e senza smorzamento per stimare $\omega_0$ e $\gamma$.
Per stimare $\omega_0$ abbiamo stimato il periodo dal grafico  confrontando punti di stessa ampiezza e per stimare $\gamma$ abbiamo preso dei punti di ampiezza massima e visto quanto tempo ci metteva per smorzarsi di $e$.
Con questo metodo abbiamo ottenuto:
\begin{equation}
\omega_0=
\end{equation}


\subsection{Analisi dei dati}



\section{Conclusione}
 Il valore del $\chi^2$ per le due rette di fit conferma  la validità del modello fisico utilizzato per il comportamento della molla. La stima di $g$ è in ottimo accordo con il valore dichiarato per Pisa di $g=9.807$ che rientra nell'errore della nostra misura. Il fatto che il $\chi^2$ sia così basso nel secondo fit può voler dire che abbiamo sovrastimato l'errore e in realtà i nostri strumenti hanno una precisione maggiore di quella attribuita.
\end{document}






